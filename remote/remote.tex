%-----------------------------------------------------------------------------
\section{Fundamentals}

Definitions: The \emph{local} computer is the one you are working on and from
which you connect. The \emph{remote} computer is the one you connect to, over a
network.

\subsection{I want to connect to a remote computer}

You will use the \lstinline{ssh} program (``secure shell'') to connect to a
remote computer. The most straightforward way is to run the command

\begin{lstlisting}
ssh username@address.of.remote.computer
\end{lstlisting}

The program will then ask for your remote password and open a shell on the
remote computer. The address could be a hostname such as
\cmd{server.youruniversity.edu}, or it could be an IP address such as
198.192.10.1. You can leave out the username if the remote username is the same
as the local username.

\subsection{I want make it easier to connect to a remote computer that I use a
lot}

Typing the whole \cmd{ssh} command with the address and username every time is
tedious. There is a very simple way to expedite this.

In your home directory there will be a directory called \cmd{.ssh}. This
directory may contain a file called \cmd{config}, which can be used (among
other things) to set up shorter aliases for connections you use a lot. If this
file does not exist, you can create it.

Let's say your username is \cmd{galileo} and you need to connect to a remote
machine called \cmd{lab.galilei.edu}. You can put the following in your
\cmd{.ssh/config} file:

\begin{lstlisting}
Host lab
    HostName lab.galilei.edu
    User galileo
\end{lstlisting}

On the first line \cmd{lab} is an alias, which you can choose yourself. With
this setup done, typing \cmd{ssh lab} is equivalent to typing \cmd{ssh
galileo@lab.galilei.edu}.

\subsection{I want to transfer a file to or from a remote computer}

The command to do this is \cmd{scp}, short for ``secure copy''. It works very
much like regular \cmd{cp}, except you can add an identifier for a remote
computer to the file paths, separated by a colon. This identifier can either be
a full \cmd{username@address} specification or an alias defined in
\cmd{.ssh/config}. Like \cmd{ssh}, \cmd{scp} will ask for your remote password.
With the above definitions, these two commands both copy the file
\cmd{test.txt} from the home directory on the remote computer to the local
working directory:

\begin{lstlisting}
scp galileo@lab.galilei.edu:test.txt .
scp lab:test.txt .
\end{lstlisting}

Copying a local file to the remote works in a similar way. To copy
\cmd{example.png} from the local working directory to the \cmd{\~/public_html}
directory on the remote, you can write

\begin{lstlisting}
scp example.png lab:public_html/
\end{lstlisting}

\section{SSH keys}

\subsection{I want to stop \cmd{ssh} asking for my password all the time}

Passwords are one way to establish that a user who requests an SSH connection
really has the right to log in. But they are not the only way. The smooth way
to use SSH is to \emph{exchange keys}, which proves to the remote machine that
the user requesting access is your user account on your local machine.

In a way you extend your authentication on your local machine to your remote
machine by giving your local user account automatic access to the remote. This
is very useful for connections you use a lot, especially when copying files
with \cmd{scp}. It's also absolutely necessary if you want to use \cmd{sshfs}
(see below).

In public-key cryptography you generate a pair of two keys. The \emph{public
key} can only encrypt text, which only the corresponding \emph{private key} can
decrypt. To communicate, two people can share public keys with each other over
an unsafe channel. Then they can send each other encrypted messages that only
the receiver can decrypt. Anyone who is eavesdropping only gets the public
keys, so cannot decrypt any messages.

If you have a public key from a person you know, you can use it to confirm
their identity e.g. by encrypting a message that only you know, and asking them
send it back to you.

\subsubsection{Generating a key pair}

To generate a key pair for SSH, run the command \cmd{ssh-keygen}. This will ask
some questions, but you can press enter to give the default value to all of
them\footnote{Note that you can set a passphrase for the keys, but then ssh
will ask for this passphrase every time you use the key, which is what we want
to avoid here.}.

The public key will be in \cmd{.ssh/id_rsa.pub}. This is what you give to the
remote server to establish your identity. The contents of this file start with
``\cmd{ssh-rsa}'', then have a long string of random characters, and finally
somthing like the hostname of your local computer.

\subsubsection{Setting up the identification on the remote}

Log into the remote server using your password. Then edit (or create) a file in
your home directory on the remote server, called \cmd{.ssh/authorized_keys}.
Copy the contents of \cmd{id_rsa.pub} into this file. Be sure to copy the
entire line and keep it on line in the file. The \cmd{authorized_keys} file can
contain many keys, each on its own line.

Now, the next time you open an SSH connection to that remote, it should log you
in directly without asking for the password.

\subsubsection{An easier way to copy the key}

Many computers these days have a program called \cmd{ssh-copy-id} installed. In
this case, after generating your key pair with \cmd{ssh-keygen}, you can run
\cmd{ssh-copy-id <remote's address>}, and it will ask for your password and
copy the key automatically to the remote's \cmd{authorized_keys} file.


\section{Keeping jobs going}

\subsection{I want to start a job up on a remote computer and make sure it
keeps running even if I am disconnected}

\cmd{nohup} is your friend here. If you start you job with \cmd{nohup}, then
even if you close the terminal window, kill your \cmd{ssh} session, etc., it
will keep running. Note that you should redirect the output (including
\cmd{stderr})

\begin{lstlisting}
nohup myjob &> out &
\end{lstlisting}

For some commands, especially those run with MPI, you also want to redirect
input, such as:

\begin{lstlisting}
nohup mpiexec -n 4 ./myjob &> out < /dev/null &
\end{lstlisting}

\subsection{I want to be able to reconnect to a job that was running in a
terminal that got disconnected}

GNU screen is the solution---this sets up a special terminal session that can
be detacted from your terminal (i.e., you can logout, close the window, etc.),
and then restore the session when you log back in. For \cmd{screen}, commands
are entered by first typing \cmd{^-a}, where \cmd{^} is the control key.

\section{Advanced SSH tricks}

\subsection{I want to get an SSH connection through a gateway server into a
private network (SSH tunnels)}

For security reasons, many institutions have an internal network, and computers
which are only accessible from within that network. Sometimes they also have
gateway servers, which provide an interface between the Internet and the
internal network. SSH tunnels provide a way to set up connections directly from
your computer to computers on the internal network transparently. Needless to
say, you need to have access credentials to both the gateway and the internal
computer.

Let's say your gateway server's address is \cmd{gateway.galilei.edu}, and the
internal server inside the network is \cmd{bunker.galilei.edu}. Your username
on both computers is \cmd{galileo}. The easiest way to set up SSH tunneling is
to put something like the following in \cmd{.ssh/config}:

\begin{lstlisting}
Host bunker-tunnel
    Hostname gateway.galilei.edu
    LocalForward 33001 bunker.galilei.edu:22
    User galileo
Host bunker
    Hostname localhost
    Port 33001
    User galileo
\end{lstlisting}

The first Host definition sets up \cmd{bunker-tunnel} so that it creates a
tunnel from port 33001 on your local computer, to port 22 on
\cmd{bunker.galilei.edu}, \emph{through} the gateway server. Port 22 is the
port used for SSH connections by default. The way you should read this
definition is ``Make a connection to the gateway server, and there make a
tunnel that goes from the local computer to the internal server''.

\cmd{localhost} always points to the local computer, so the second Host
definition sets up a connection to port 33001 on the local computer. The trick
is that when the tunnel connection is open, this port is one end of the tunnel
which leads to port 22 on \cmd{bunker.galilei.edu} instead.

To use the tunnel, first open the \cmd{bunker-tunnel} connection. This
establishes the tunnel, and the \cmd{bunker} connection works. The \cmd{-N}
option to the \cmd{ssh} command stops it from opening a shell on the gateway
server. Instead it only sets up the tunnel.

\begin{lstlisting}
ssh -N bunker-tunnel &
ssh bunker
\end{lstlisting}

All the commands we describe in this chapter (\cmd{ssh}, \cmd{scp} and
\cmd{sshfs}) work with the \cmd{bunker} alias normally, as long as the tunnel
remains open.

After finishing your work on \cmd{bunker} and closing the second SSH session,
run \cmd{fg} to bring the tunnel SSH session to foreground, and close it with
\cmd{ctrl-c}.

The port number 33001 we chose above is arbitrary, but it's a good idea to avoid
using a port number that's used by other programs. The 33000 range is a good
choice of numbers.

\subsection{I want to mount a remote directory onto the local filesystem over
SSH}

Copying files around with \cmd{scp} is all fine, but what if you could remotely
mount a directory on a remote computer so that for all intents and purposes it
behaves like a directory on your local computer, except all the data transfer
happens over SSH? The answer to this is \cmd{sshfs}.

