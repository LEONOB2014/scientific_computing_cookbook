\documentclass[11pt]{book}

% include figures
\usepackage{epsfig,wrapfig}

% prefer PDF to PNG
\DeclareGraphicsExtensions{%
  .pdf, .png}

%\graphicspath{}

\usepackage[T1]{fontenc}

% AMS symbols
\usepackage{amsmath,amssymb}

% cancel
\usepackage{cancel}

% Palatino font (and math symbols) -- looks nicer than the standard
% LaTeX font
\usepackage{mathpazo}
%\usepackage{helvet}
%\usepackage{lmodern}   % better computer modern sans needed with fontenc
\usepackage{gillius}

% URLs (special font for monospace)
\usepackage{inconsolata}
%



% margins and paper size -- this needs to come before fncychap
% see http://stackoverflow.com/questions/3729099/latex-paper-size
\usepackage[top=0.75in,
            bottom=0.75in,
            inner=0.9in,
            outer=0.65in]{geometry} 
\geometry{papersize={7in,10in}}


% this needs to be done before the fncychap style, since that will 
% redo the chapter stuff
\usepackage{sectsty}
\allsectionsfont{\sffamily}


\usepackage[PetersLenny]{fncychap}
\ChNameVar{\fontsize{16}{18}\usefont{OT1}{phv}{m}{n}\selectfont}
\ChNumVar{\fontsize{60}{62}\usefont{OT1}{ptm}{m}{n}\selectfont}
\ChTitleVar{\Huge\bfseries\sffamily}
\ChRuleWidth{1pt}

% part page style                                                                                  
% see http://tex.stackexchange.com/questions/6609/problems-with-part-labels-using-titlesec         
\usepackage{titlesec}

\titleformat{\part}[display]
   {\bfseries\sffamily\Huge\filcenter}
   {{\partname{}} \thepart}
   {0em}
   {\hrule}




% hyperlinks -- load after fncychap
\usepackage{hyperref}
\usepackage[hyperpageref]{backref}
\renewcommand*{\backref}[1]{}% for backref < 1.33 necessary
\renewcommand*{\backrefalt}[4]{%
\ifcase #1 %
  No citations.%
\or
  (Cited on page #2)%
\else
  (Cited on pages #2)%
\fi
}

% back references


% color package
\usepackage{xcolor}
\definecolor{mygray}{gray}{0.5}


% custom hrule for title page
\newcommand{\HRule}{\rule{\linewidth}{0.125mm}}

\newcommand{\pyro}{{\sf pyro}}
\newcommand{\git}{{\tt git}}

\newcommand{\ddx}[1]{{\frac{{\partial#1}}{\partial x}}}




% don't make the chapter/section headings uppercase.  See the fancyhdr
% documentation (section 9)
\usepackage{fancyhdr}
\renewcommand{\chaptermark}[1]{%  
\markboth{\chaptername
\ \thechapter.\ #1}{}}

\renewcommand{\sectionmark}[1]{\markright{\thesection---#1}}


% don't put a header on blank pages, see
% http://www.latex-community.org/forum/viewtopic.php?f=4&p=51559
% change ``plain'' to ``empty'' to eliminate the page number
\makeatletter
\renewcommand*\cleardoublepage{\clearpage\if@twoside
\ifodd\c@page\else
\hbox{}
\thispagestyle{empty}
\newpage
\if@twocolumn\hbox{}\newpage\fi\fi\fi}
\makeatother




% skip a bit of space between paragraphs, to enhance readability
\usepackage{parskip}


% captions
\usepackage{caption}
\renewcommand{\captionfont}{\footnotesize}
\renewcommand{\captionlabelfont}{\footnotesize}
\setlength{\captionmargin}{3em}


%-----------------------------------------------------------------------------

% license
\usepackage{ccicons}

% computer keyboard symbol
\usepackage{marvosym}


% for dotted lines in the matrics/arrays
\usepackage{arydshln}



% fonts for TOC, list of figures, etc
\renewcommand*\listfigurename{\bf\textsf{List of Figures}}
\renewcommand*\contentsname{\bf\textsf{Table of Contents}}



% short table of contents
\usepackage{shorttoc}


% pack more figures on a page
\usepackage{float}
\renewcommand\floatpagefraction{.9}
\renewcommand\topfraction{.9}
\renewcommand\bottomfraction{.9}
\renewcommand\textfraction{.1}   


% MarginPars                                                                    
\setlength{\marginparwidth}{0.75in}
\newcommand{\MarginPar}[1]{\marginpar{\vskip-\baselineskip\raggedright\tiny\sffamily\hrule\smallskip{\color{red}#1}\par\smallskip\hrule}}




\usepackage[many]{tcolorbox}

\newtcolorbox{mybox}[1][]{
    width=\textwidth,
    arc=0mm,
%    auto outer arc,
    boxsep=0cm,
    toprule=1pt,
    leftrule=0pt,
    bottomrule=1pt,
    rightrule=0pt,
    colframe=black,
    colback=white,
    fontupper=\centering\fontsize{40pt}{40pt}\sffamily,
    breakable,
    nobeforeafter,
    enhanced jigsaw,
    opacityframe=1.0,
    opacityback=0.7
}


\pdfpageattr{/Group <</S /Transparency /I true /CS /DeviceRGB>>} 


% special colors and definitions for framing source code listings
\usepackage{listings}

\definecolor{AntiqueWhite3}{rgb}{0.804,0.753,0.69}
\definecolor{DarkerAntiqueWhite3}{rgb}{0.659,0.635,0.612}
\definecolor{orange}{rgb}{1.0,0.65,0.0}

\lstset{%
  keywordstyle=\color{blue}\ttfamily,%
  commentstyle=\itshape\color[gray]{0.5},%
  mathescape=true,%
  basicstyle=\small\ttfamily,%
  %frameround=fttt,%
  frameround=ffff,%
  %frame=shadowbox,%
  frame=single,%
  rulesepcolor=\color{DarkerAntiqueWhite3},%
  backgroundcolor=\color{AntiqueWhite3},%
  emph={load,add_slice,save}, emphstyle=\color{blue},%
  emph={[2]In}, emphstyle=[2]\color{yellow},%
  emph={[3]Out}, emphstyle=[3]\color{orange},%
  xleftmargin=1em,
  xrightmargin=1em,
  mathescape=false}



\input symbols

\begin{document}


\frontmatter

\begin{titlepage}

\ \\[2.5in]
\begin{center}
\begin{mybox}[]
\vskip 3mm
{\Huge Scientific Computing Cookbook} \\[-0.5em]
{\Large part of the Open Astrophysics Bookshelf}
\end{mybox}
\end{center}

\vfill

\begin{flushright}
\today
\end{flushright}

\end{titlepage}

\pagestyle{plain}

\null \vfill 

\noindent \ccCopy\ 2015, the Open Astrophysics Bookshelf project \\
\noindent document git version: \input git_info.tex $\ldots$

\noindent the source for these notes are available online (via \git): 
\url{https://github.com/Open-Astrophysics-Bookshelf/scientific_computing_cookbook}

\noindent \ccbyncsa \\
\noindent This work is licensed under the Creative Commons
Attribution-NonCommercial-ShareAlike 4.0 International (CC BY-NC-SA
4.0) license.


\clearpage


\shorttoc{Chapter Listing}{0}

\clearpage


\setcounter{tocdepth}{2}
\tableofcontents


\clearpage

\chapter*{preface}
\chaptermark{preface}
\addcontentsline{toc}{chapter}{preface}

\input preface/preface.tex

\clearpage

\pagestyle{headings}

\renewcommand{\chaptermark}[1]{%  
\markboth{\chaptername
\ \thechapter.\ #1}{}}

\renewcommand{\sectionmark}[1]{\markright{\thesection---#1}}

% put the git hash on the first page of each chapter -- see section
% 7 of the fancyhdr docs to see how to override the plain
% style
\fancypagestyle{plain}{%
\fancyhf{} % clear all header and footer fields
\fancyfoot[C]{\thepage} % except the center
\fancyfoot[L]{\scriptsize git version: \input git_info.tex $\ldots$}
\renewcommand{\headrulewidth}{0pt}
\renewcommand{\footrulewidth}{0pt}}


\mainmatter

%-----------------------------------------------------------------------------
\part{The UNIX Shell}

\chapter{Basic operations}

\input UNIX/unix.tex


\chapter{Advanced Scripting}


\chapter{Working Remotely}



%-----------------------------------------------------------------------------
\part{Programming}

\chapter{Python}

\chapter{Fortran}

\chapter{Parallel Programming}



%-----------------------------------------------------------------------------
\part{Plotting and Data Analysis}




%-----------------------------------------------------------------------------
\part{Software Engineering Practices}

\chapter{Version Control}

\chapter{Testing}

\chapter{Reproducibility}


\backmatter

%\addcontentsline{toc}{chapter}{References}
%\bibliographystyle{plain}
%\bibliography{refs}

\end{document}
