%-----------------------------------------------------------------------------
\section{Picking up where we left off}

Recipes that help us know what source we used, etc.\ when running
a simulation or producing other output.


\subsection{I want to store the git hash of my source code in all the output files I produce}

If you code is in version control with git, then the git hash uniquely
identifies the version of the source you are using.  By storing this
hash in your output, you always know what version of the source was
used to get your results.

To accomplish this, we create a special source code target in our
makefile that is produced at compile time.  This source code will
contain a function whose job it is simply to return the git hash.
Since this is compiled into the code, we can then store the hash in
any of our output.  Here's an example that writes a Fortran
subroutine, but it is easily adapted to other languages.

First we create a simple shell script that will output the
Fortran file with a function {\tt githash()} that returns
the hash.  We use the git command: {\tt git rev-parse HEAD}
to find the hash of the current head.  It is assumed here
that {\tt make} is run from within the git-controlled
source tree.

\begin{lstlisting}[language={bash},upquote=true]
#/bin/sh

HASH=`git rev-parse HEAD`

cat > githash.f90 <<EOF
module gitstuff

  implicit none

  integer :: HASHLEN=${#HASH}

contains

  function githash() result (hash)
    implicit none

    character (len=HASHLEN) :: hash
    hash = ``${HASH}''

    return
  end function githash
end module gitstuff
EOF  
\end{lstlisting}

Next we need to call this when we compile, by creating
a special target for the source file, {\tt githash.f90}.
The following {\tt GNUmakefile} will compile two source
files, {\tt main.f90} and {\tt githash.f90}, specifying
the dependency between them.

\begin{lstlisting}[language={make}]
# compiler
FCOMP := gfortran

# list of source files
f90sources += main.f90 githash.f90

# list of resulting object files
f90objects = $(f90sources:.f90=.o)

# main target to build
all: main

# inter-file dependencies
main.o: githash.o

# githash.f90 special target
# note that we remove after compiling
githash.f90:
     ./makegithash.sh

# rule to make a .o from a .f90 file
%.o: %.f90
     $(FCOMP) $(FFLAGS) -c $<

# rule to build the main target
main: $(f90objects)
     $(FCOMP) -o main $(f90objects)
     rm -f githash.f90
\end{lstlisting}

We can now call our {\tt githash()} function from any routine
and store the resulting string in our output, for example
in our {\tt main} program:

\begin{lstlisting}[language={fortran}]
program main

  use gitstuff

  implicit none

  print *, ``running code version: ``, githash()

end program main
\end{lstlisting}
  



